With the statistical and analytic machinery in place, the results are 
rather straight-forward. However, the results are also rather striking. 
First, as discussed in Section \ref{sec:Methods}, we computed the mixed model
given by \texttt{count $\sim$ phase + facet + (1|date)} for 266 different date
partitions determining which date was assigned to which phase. The first
partition date ranged between September 26 and October 9, 2016. The
second partition date was between October 27 and November 14, 2016.
The phase was 1 if the date was before the first partition date, 2 if the 
date was on or after the first partition date and before the 
second partition date, or 3 if the observation date was on or after the second 
partition date.


\begin{figure}
  \centering
  % \vspace*{-.55in}
  \hspace*{-.25in}
  \includegraphics[width=1.25\textwidth]{figures/AIC_dates-FIG2.pdf}

\caption{
  Akaike information criterion (AIC) for models of \texttt{count $\sim$ phase + facet + (1|date)}
  for 266 different phase assignments. The phase was 1 if the date was before
  the first partition date, 2 if the date was on or after
  the first partition date and before the second partition date, or
  3 if the observation date was on or after the second partition date.
}
\label{fig:AICs}
\end{figure}


\begin{figure}
  \centering
  % \vspace*{-.55in}
  \hspace*{-.25in}
  \includegraphics[width=1.25\textwidth]{figures/rel_likelihood_dates-FIG2.pdf}

\caption{
  Relative likelihood of models with identical formula, but different dates
  used to partition the data into three ``phases''. The phase 
  was 1 if the date was before
  the first partition date, 2 if the date was on or after
  the first partition date and before the second partition date, or
  3 if the observation date was on or after the second partition date.
}
\label{fig:relative-likelihoods}
\end{figure}

Recall that a model with relatively higher AIC means more information was lost by
that model. So, a lower relative AIC is better. However, miniscule differences
in AIC cannot be considered conclusive. To select the model that best fits the
data, which in turn selects the dates that best correspond to the locations
of the phase transitions, we first compute the AIC of all candidate models, 
The results of this calculation for all candidate models that include phase
are shown in Figure \ref{fig:AICs}. To determine the probability that 
another model will not minimize information loss better,
the model with the minimum AIC is compared to all $N-1$ other models using
the relative likelihood, $\mathcal{L}_i$, where $i = 1, \ldots, N;~i \neq i_{min}$,
where $i_{min}$ is the model index with minimum AIC. $\mathcal{L}_i$ was defined
in Equation \ref{eq:relative-likelihood}. 

The pair of partition dates with the smallest AIC was October 10 and
November 8, with an AIC of 949. We then calculated $\mathcal{L}_i$ for all 
$N-1$ other models, 
with the results shown in Figure \ref{fig:relative-likelihoods}. At first,
we might be discouraged by the results. While the we do find a minimum
AIC that is significantly less than the null model 
($\mathrm{AIC}_{\mathrm{null}} = 955.63;~\mathcal{L}_{\mathrm{null}} = .046$),
there are many candidates with high relative likelihoods. 
There is a explanation that does not imply that our method for
identifying the location of a phase transition is not working. What if,
instead, our procedure is identifying an area of mixed phase, or a transient
phase to use more of a dynamical systems term. 
