Does metaphorical violence usage increase when debates are ongoing? The data
will show that yes, usage does increase when looking at daily usage between September and November,
2016. To build up to that conclusion, we will first look at the performance
of a number of models with various hypothetical dates when the increase
began and ended. The daily timeseries of frequency of metaphorical violence
usage is shown in Figure \ref{fig:timeseries-fit}. Along with the timeseries
of frequency are the mean frequencies during each of the two time periods of
interest: the period of normal or base usage, and the period of increased usage,
in the center straddling the occurrence of the three debates. Recall that the
dataset under consideration is a daily timeseries of frequencies of meatphorical 
violence usage.  

\begin{figure}[h!]
  \centering
  \includegraphics[width=\textwidth]{figures/rel-lik-zoomout.pdf}
  \caption{Relative likelihood of models with various start
  and end dates. The y-axis represents the hypothetical start date of the state of
  elevated metaphorical violence usage. The x-axis represents the hypothetical
  end date of the elevated state. 
  The models with highest relative likelihoods are colored darker. Most of the
  pairs of dates are not likely, which is to be expected.}
  \label{fig:rel-lik-zoomout}
\end{figure}

To determine
the exact beginning and ending of the period of increased usage, I compared
the AIC of the model fit to all the hypothesized start and end dates. We compare
two model fits by the relative likelihood, obtained by calculating
the exponential of the halved difference between the minimum AIC and another
AIC (Equation \ref{eq:rel-lik}).
There is only a small region of the relative likelihood map where the relative
likelihood is significantly greater than 0 (Figure \ref{fig:rel-lik-zoomout}).
Zooming in on this region, we see that the model that captures the most information
is the one where the elevated period begins September 25 and ends October 26
(Figure \ref{fig:rel-lik-zoomin}).
The start date of the elevated state (the y-axis of Figure \ref{fig:rel-lik-zoomin})
is well-defined. There is no other date that results in a likely model. 
However the end date is less certain. 
The end dates are the x-axis of Figure \ref{fig:rel-lik-zoomin}.
Plus or minus one day results in a model that is .71 and .79 times as likely
to capture more information. There is a window of six days where an alternative
ending date of the elevated usage results in highly likely models. 

\begin{figure}[h!]
  \centering
  \includegraphics[width=\textwidth]{figures/rel-lik-zoomin.pdf}
  \caption{Zoomed-in detail of the most likely start and end dates of increased
  metaphorical violence usage are most likely. The y-axis is the hypothetical
  start date of increased usage, and on the x-axis are hypothetical end dates 
  of metaphorical violence usage. The most likely start and end date are
  September 25 and October 26. The start date is definitively identified, while
  the end date is less certain.}
  \label{fig:rel-lik-zoomin}
\end{figure}



With the best start and end date of the elevated state of usage established,
we may now calculate the resulting means for each state. Normal usage of 
metaphorical violence is about 1 use per episode per day. The elevated usage 
is 2.7 uses per episode per day. The relative likelihood of the null model,
where there is only one state of usage, is much less than 0.05. For those
who prefer the frequentist approach, the p-value of the linear fit is
nearly zero. 

\begin{figure}
  \centering
  \includegraphics[width=\textwidth]{figures/timeseries-fit.pdf}
  \caption{Daily timeseries of frequency of figurative violence usage. The
  mean frequency is overlaid. The normal frequency is 1.0, while the mean
  frequency in the elevated region is 2.7. As revealed by the
  relative likelihood analysis, the start date of the elevated state is 
  September 25, and the end date of the elevated state is October 26.}
  \label{fig:timeseries-fit}
\end{figure}

