People rarely think twice about metaphorical violence when talking about 
political debate, strategy, and elections. But we should think twice because
such talk has real-world consequences. Before we can think twice about how we
talk about politics we need to know exactly what is said about politics. What language
we use depends on what is going on around us, so we must also know when our
language is apt to change and what might be driving that change. In a corpus analysis
of metaphorical violence on cable news, I show that metaphorical violence increases
around the the presidential debates 
preceding the United States presidential election of 2016. In support of this
analysis, I also present two novel, reusable, and open-source software packages. 
The first provides free, programmatic access to cable news archives hosted
by the Internet Archive. The second is Metacorps: a data model, web application,
and analytical pipeline for efficient, collaborative metaphor coding across corpora. 
These two software packages feed a statistical analysis that pinpoints when metaphorical
violence usage increases, and quantifies the level of increase, under the
influence of the presidential debates. This study increases our understanding of the
dynamics of metaphorical violence in the realm of politics. Such understanding
is important for the design of behavioral studies, 
which in turn will lead to a healthier democracy through more inclusive and 
effective political communication.
