Language, thought, and behavior are inextricably linked. In certain extreme
cases, like an army general giving orders, the links between the three clear. 
In other cases, like politics, the chain of events is less clear. Did some
political behavior cause thought, which produced a certain utterance? Was a
conceptual cascade triggered by something that was said? Did someone not vote
because of what they heard on television? Language has the ability to reinforce
or alter the relationships between concepts in our mind. Similarly, the relationship
between concepts in our mind drive how we talk about things. Not only do
relationships between concepts drive utterances, but the relationship between
speaker and audience also drives what is said. In many cases, pragmatic choice,
what is said and how, is made at an unconscious level. However, that does not mean
that the way a particular issue is commonly framed is optimal. In fact, to 
communicate as effectively as possible with the best outcomes, we need to check
whether what we say fits with our goals and values. We need to evaluate the
conceptual structures that have become established in our own mental space and
the mental space of society as a whole. The first step towards evaluating
our concepts and language is understanding what we say and why we say it. 
This paper examines a specific and ubiquitious conceptual structure that may
pose challenges for effective democratic decision-making: the general conceptual metaphor
\textsc{politics is a physical conflict}. I show that this results in metaphorical
violence usage increasing on cable news shows around the days of the 2016 
presidential debates. Furthermore, I pinpoint the exact date increased metaphorical
violence usage on cable news begins and the exact date it ends. This provides
the much needed starting point for us to ask whether or not such language
use is healthy for democracy.

While metaphorical violence in politics may be normal, we should not underestimate its
importance or assume it has no real-world consequences. 
Its effects are more than increased rhetorical power. 
When more aggresive people read political messages containing mild metaphorical violence, 
their support for political violence increases. Less aggressive people tend to
lose faith in government efficacy when they are exposed to metaphorical violence
in political messaging \cite{Kalmoe2014}. 
In an extreme but important example of this, Representative
Gabrielle Giffords' Arizona congressional district was overlaid with a gun sight. 
She was among 20 house democrats who were singled out for voting for the 
Affordable Care Act. The map was produced by Sarah Palin's political action 
committe SarahPAC. Palin herself announced the campaign to retake the house 
seats, tweeting ``Commonsense Conservatives \& lovers of America: `Don't Retreat,
Instead - RELOAD!' Pls see my Facebook page.'' This tweet was published March
23, 2010. In January, 2011, following the 2010 congressional elections, 
Representative
Giffords was shot at a ``Congress on your Corner'' event at a Safeway in 
Tuscon, Arizona. She miraculously survived a headshot wound, but six others
were killed including a Federal judge and a nine-year-old girl. Fifteen were
wounded. As Giffords herself said in an interview on MSNBC nine months before
the attack ``we've gotta realize that there are consequences to that action...we have to work out 
our problems by negotiating and working together.''

Elections are an improvement over violent struggles for power. However, this
has not eliminated all political violence, and it certainly has not elminated
metaphorical violence from discourse about debates and elections. 
Metaphor is an essential human cognitive capability. We understand abstract
concepts, like a debate, in terms of embodied experience, like a fight. Possibly
because of the fact that elections replace violent transfers of power, 
English-speaking Americans are inundated with metaphorical violence describing
political messaging and debate strategy. Language is often underspecified, 
meaning many details are left out. Metaphorical violence leaves individual 
listeners to fill in details
themselves. For example, on Fox News' \textit{The Kelly File}, Megyn Kelly
gave Donald Trump advice on how to ``fight'' against Hillary Clinton in the
second presidential debate: ``Punch back. Don't defend everything she lobs your
way, don't be on defense. But punch back. Punch at her, and punch relentlessly,
actually, as the night goes on.'' We might hear something like this at home,
at a restaurant, in a waiting room, or in the airport. But taken out of context,
we can't know that Megyn Kelly is talking about a debate. She might well be
talking about Ronda Rousey's strategy in an upcoming Ultimate Fighting Championship
cage match.

What is metaphor exactly, and how does it work in general? The ability to 
think and communicate with metaphor is a basic human cognitive ability.
\textit{Conceptual metaphor} is used to relate two or more 
concepts.  Typically one of the concepts is more abstract and the other is
more concrete, or \textit{embodied}, meaning it's something we've physically
experienced. We understand the abstract concept in terms of the more embodied
concept. For example, a recurring conceptual metaphor in political discourse
is \textsc{a debate is a fight}. Conceptual metaphors by convention are written
in small caps. The first concept is called the target domain and the second 
concept is called the source domain. Elements of the source domain, the embodied
domain, are mapped on to the target domain. In \textsc{a debate is a fight}, 
the rather abstract concept of a debate is understood in terms of particular
elements of a fight, such as heightened emotions, the importance of
strategy, and the necessity of endurance. Importantly, metaphor also hides 
certain, sometimes fundamental, elements of the target domain. In understanding
a presidential debate as a fight, we ignore the ultimate purpose of such debates:
to identify which candidate's plans would most benefit us as individuals and
the nation as a whole. Framing debates as a fight assumes that there must be
a winner and a loser, with no other redeeming features of the outcome. While
a fight may be decided by the quality of a fighter's strategy, the strategy is
hidden from most observers. On the other hand, in a debate we should pay careful
attention to the reasoning and rationale behind the contestants' statements.
We should base our voting behavior not on who won or lost in the sense of a
fight, but instead who presented better policy with more convincing arguments.
Such conceptual relationships are expressed as linguistic instances of metaphor,
which are the behavioral data collected in this study. 

There are real consequences to the exaggeration of certain elements of debate and
hiding others.  As such, this work expands our understanding of how
metaphorical violence is used on cable news. Cable news is watched in millions
of households every night when it's broadcasted, watched by millions more online, 
and showed at restaurants, airports, and waiting rooms across the country.
From cable news alone, we are bombarded with metaphorical violence. Cable news
also provides us with convenient, if inexact, representations of the 
political left, center, and right, by different channels. MSNBC is the 
left-leaning channel, CNN represents the center, and Fox News represents
right-wing ideologies \cite{Bode2014}. Cable news as a source for corpora is 
especially enlightening
because it must satisfy a number of communicative requirements. First, it must present some
trustworthy semblance of the true nature of current events. Secondly, it must
communicate current effects effectively to its audience, meaning cable news
channels must adopt language that is easily understood by its viewers. Third,
cable news channels must reinforce conceptual relationships deemed profitable---this
may include perpetuating political biases or avoiding negative coverage of
sponsors. People form emotional attachments to the anchors, even considering them
family members. \citeA{Hochschild2016} interviewed a conservative person from
Louisiana who described various Fox News anchors: 
"'Bill O'Reilly is like a steady, reliable dad. Sean Hannity is like a difficult uncle who rises to anger too quickly. Megyn Kelly is like a smart sister.'" We need to think critically
about what impact all this has on democratic participation.

Very little systematic work has been done examining the dynamics of metaphor use over time,  
especially on a daily timescale, as presented here, and especially regarding 
metaphorical violence about politics on cable news. 
Cable news data itself is not straight-forward to 
acquire. The Vanderbuilt cable news database is one resource, but it is 
not publically available---one's institution must pay for access. 
The software tools I present here, combined
with the immense database hosted by the Internet Archive, 
provide a novel, open, and programmatic way to access large amounts of cable news data. 

Metaphor has been extensively studied as it relates to politics. 
However, metaphor is typically studied without any explicit
dependence on what is going on inside or out of the discursive environment
that produced the metaphor.  
The work presented here expands our knowledge of
how metaphor use changes over time in response to political events \cite{Gibbs1997}. 
We know, for example, that metaphors for war hide the subleties and realities of armed
conflict \cite{Lakoff1991}. \citeA{Lakoff2004, Lakoff2008}, and \citeA{Lakoff2012} 
identify existing conceptual metaphors and their linguistic instantiations. They
argue that the two major political divisions
of the United States, liberal and conservative, each have their own 
specialization of
the general conceptual metaphor \textsc{the nation is a family}. For liberals
the family is a ``nurturing parent family'' where the parents' genders are
unimportant and children are loved first and disciplined second; discipline is
never violent. On the other hand, conservatives are supposed to conceptualize
the nation as a ``strict-father family.'' In this family, parents hold to 
rigid gender roles, children are punished with violence, homosexuality and
abortion are both threats to future generations of strict-father families.
This series of work is aimed at liberals and progressives with the goal of 
making their language more convincing to the general public. So, implicitly, 
language is assumed to be changable. But here we make this assumption explicit
and investigate the direct relationship between metaphorical violence and 
the political event of United States presidential debates.

% We know some facts about why metaphor is used in politics
% Another framework to mention is that of 
% \citeA{Charteris-Black2009}, which claims the ultimate goal of metaphor in
% politics is ``forming legitimacy.'' This is done through four main types of 
% metaphors: ``metaphors that establish ethical integrity,'' ``metaphors that
% communicate political arguments,'' ``metaphors that heighten emotional impact,''
% and ``metaphors that communicate ideology and political myth.'' These frameworks
% are useful, but do not bring us closer to understanding the dynamics of
% metaphor use over time. When certain metaphors are used, and why are particular
% metaphors are used? The aim of this work is to 
% find explicit evidence of the connection between the presidential debates of 
% the 2016 United States elections and the usage of metaphorical violence on 
% cable news. In the process, I built software that enables more efficient
% large-scale analysis and annotation of metaphor in a corpus. I also introduce 
% a statistical method for finding when and how general language use changes, 
% which I apply to this particular case of finding when and how much 
% metaphorical violence usage increases.

Using cable news transcripts from a three-month period, September to November, 2016, 
I show that metaphorical violence usage
increased on cable news around the time of the presidential debates. By 
marking each date as either ``normal'' or ``elevated,'' as in normal levels
of metaphorical violence usage or elevated levels of metaphorical violence
usage, and fitting a simple linear model, I found that metaphorical violence
usage did in fact increase from about 1 instance per episode per day to 
2.7 instances per episode per day. This was for the top two shows on each network, MSNBC,
CNN, and Fox News. Furthermore, by varying which days were coded ``normal''
and ``elevated,'' I pinpointed the exact date the increase in usage started and
ended: the elevated usage began September 25 and ended October 26. For reference,
the first debate was September 26 and the final of three debates was October
19. So, increased usage of metaphorical violence nearly exactly coincides with
the first debate, and continues after the last debate for about a week. However
the exact end date is less exactly determinable, possibly due to the election
on November 8.
