In the previous sections, a method was developed and applied that identifies
phase transitions in uses of metaphor. Such an approach supports the 
proposal it was inspired by: metaphor is an emergent property of a complex
dynamical system \cite{Cameron2006}. The application of the method revealed 
the method also detected transience, an unexpected feature of the analysis.

This result is not just of theoretical importance, however. Anthropologist
Arlie Russel Hochschild recently published a book \textit{Strangers in their
Own Land} which carefully studied the increasing animosity and 
polarization of American politics and political discourse \cite{Hochschild2016}.
To describe the problem we are all facing, Hochschild 
describes the political left and right as being on opposite sides of
an ``empathy wall'', a poignant metaphor. 
This wall is made ever higher by geographic clustering of likeminded people,
personalized media tailored to fit pre-existing political biases, and other 
factors. It seems the physiological and neurological impact of metaphorical
violence would only raise this empathy wall between fellow citizens at a time
when we desperately need it lower.

Believing that the political left is superior, cognitive linguist George Lakoff
argues in a series of books that the path to a better United States is 
one where progressives give voice to their values 
\cite{Lakoff1996, Lakoff2004, Lakoff2008, Lakoff2012}. 
This may or may not be true, 
but these works are accompanied by broad-brush claims about the persistent
mental states of questionable scientific rigor. For example, a central
claim is that ``conservative and progressive politics are organized around 
two very different models of married life: a strict father family and a 
nurturant parent family'' \cite[p.47]{Lakoff2004} But
under what circumstances does this claim hold, if the claim is even detailed
enough to hold in any meaningful sense at all? The results we demonstrate
support a claim advocated since at least \citeA{Gibbs1997} 
encouraged us to take ``metaphor out of our heads and into the cultural
world'' that 
we cannot separate our metaphors from their environment. So even if in some
broad sense Lakoff's claim is true, there are subtleties of ``progressives''
and ``conservative'' conceptual worldviews that must be in constant flux.

This work has been inspired by Hochschild's desire to understand the structure
of the empathy wall and to find ways to get over it or to lower it. By better
understanding what cultural events give rise to harmful talk that raises
the empathy wall, we can suggest better alternatives. This approach brings
more rigor to the study of pragmatics in political talk, and reveals the
deep connection between certain cultural events and our use of metaphorical
violence. It is a promising method for identifying other metaphorical 
constructions in relation to other cultural events.

Future work should first include longer timescales and other situations. For
example, we should see if there are other transitions when the national 
conventions and the preliminary debates were happening. These results 
should be compared to the same three months for the 2012 elections. There is
also much more to be done with the other data collected in the analysis of 
these three months. How does the source domain change over time? That is, 
what is the nature of the metaphorical violence? For example, does
the ratio of uses of \textit{attack} to \textit{hit} change
with each phase, for example? Would such a ratio show phase transitions
at different times? Or are some ratios fixed, with no significant phase
transition? How does tense change over time? Who is most often the subject
or object of metaphorical violence? Do subject/object relationships also
show phase transitions? Answering these and similar questions may enable us
to better prepare for and understand pragmatic ``choice'' in political talk, 
and advocate for and make pragmatic decisions that can result in less division
and more empathy in politics in general.
