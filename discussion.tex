By first pinpointing the most likely start and end dates of a hypothetical
period of elevated metaphorical violence usage, I determined that metaphorical
violence usage increased around the time the presidential debates were occurring.
This confirms what one might expect, given what we know about the role of
metaphor in cognition and communication. This is a real-time demonstration that
indeed, real-world metaphor usage changes in response to a changing cultural environment
\cite{Cameron2006, Gibbs2012a}. I have quantified this increase and determined
exactly when the change occurred.

The usage of metaphorical violence is a choice. When the decision is made to
use metaphorical violence, that decision has real-world impact. Metaphorical
violence appears to cause the most aggressive of us to support violence against politicians.
Metaphorical violence appears to cause the least aggressive of us to doubt
the efficacy of governments~\cite{Kalmoe2014}. Increased usage of metaphorical violence 
occurs just before elections. Thus, it is important to further understand the impact
that such pragmatic choice of language has on voting behavior itself. In the effort
to build a theory of the effect of metaphorical violence on voting 
behavior we have taken the first step of understanding more about the
dynamics of metaphorical violence usage.

While the results presented here do indeed increase our understanding of how
metaphorical violence works for describing political events, there is much work
that needs to be done before we can draw broad conclusions. Considering only 
three months is not nearly enough. First, more of the period surrounding the
debates and election must be examined. Other periods of formal debate should
also be analyzed, starting with the 2012 presidential debates and election. 
In both 2012 and 2016, the Republican and Democratic primary debates and
election cycle should be analyzed. This will help determine the impact of 
the presidential election itself on metaphorical violence usage. 
In all time periods, other violent words and phrases should be included in the analysis.

Beyond just exploring the frequency of metaphorical violence usage, changes in
the source domain should be examined. In other words, we need to know which 
violent words and phrases are used when. Does metaphorical violence become 
more violent as the debates go on and the election approaches? 
That is, do the metaphors become less conventional and involve
more violent words like \textit{punch}, \textit{bloody/bloodied}, or \textit{strangle}?
Furthermore, we will investigate who is most often framed as the 
aggressor and victim in metaphorical violence.
How do these relationships change over time? We might also ask
whether this increase in metaphorical violence usage is also seen across languages
within American culture, for example in Spanish-language news? What about in
other cultures? What constraints do a given language put on metaphorical violence
usage?

While these questions are big and many, there is good news: the software and
methods developed in this paper can be readily applied to all these situations.
Metaphors of any language can be analyzed using the web application Metacorps. 
Metacorps collects the subject and object of metaphorical violence. 
A team of any size can use Metacorps to safely edit multiple potential instances 
of metaphorical violence without 
worrying about data management issues like maintaining a single Excel spreadsheet 
or other document that contains potential instances of metaphor. 
The rate of coding new metaphor is
limited only by the number of analysts and the amount of time each can 
contribute to metaphor identification and analysis. Furthermore, the database
of metaphor held by Metacorps is ever-expanding, and its size will accelerate
as more people code metaphor using the system. This database of instances of
metaphor and non-metaphor can be used as labelled data for machine learning 
approaches to metaphor identification~\cite{Shutova2015, Veale2016}. Such automated
coding of metaphor is essential for large-scale analyses that include many 
sources of discourse, including online news, blogs, and social media.
Similarly, the statistical method developed here can be applied to any 
timeseries of metaphor use, or any sort of language change over time. 

The instances of metaphorical violence we collected will be used to design 
stimuli in upcoming behavioral experiments. Just like other elements of 
language comprehension, we understand metaphor through a process of simulation
\cite{Gibbs2008}. So, in the case of metaphorical violence, it would be inferred
that our fight-or-flight response should get involved, one of our most primal responses.
What impact does this have on our political reasoning and decision-making? 
We know that much of political reasoning is not done consciously, another
instance where the myth of the rational actor is destroyed~\cite{Westen}. 
I believe metaphorical
violence is a powerful force for activating deeply unconscious responses. 
Such unconscious responses drive us to reject candidates who might represent
our interests better, but we do not support them because of the reinforcement
of pre-existing biases. The presence and nature of
metaphorical framing impacts how people think about forest fires~\cite{Matlock2015} and 
climate change~\cite{Flusberg2017, Flusberg2017a}. Following these frameworks,
we can design effective instruments to further probe the effects of metaphorical
violence on political decision-making.
It is no exaggeration to say that the well-being of democracy depends on healthy
dialogue, and an aware and engaged voting populace. In the United States we have
become highly polarized in our views on the most pressing issues. 
Most of us sit on once side or another
of an ``empathy wall'' that prevents us from understanding those on the other
side~\cite{Hochschild2016}. Future behavioral experiments will aim to support
my hypothesis that metaphorical violence raises this empathy wall. 
How could we possibly empathize with
those we ``target'' in an upcoming election or with those who target us? If 
we are targeting and attacking each other, 
then we are in a state of war with each other.

Metaphor works by emphasizing certain embodied qualities of the target domain.
For example, the conceptual metapohr \textsc{a debate is a fight} highlights
the antagonistic nature of a debate, as well as the intense emotional depth
of a debate. If the debate is not a boxing match but a street fight, the 
consequence of losing could be death! While highlighting some aspects of the
target domain in terms of the source domain, metaphor also hides aspects of
the target domain. In terms of a debate, many important elements of a debate
are hidden. For example, debates are a chance to identify good ideas from all
contestants. They should be a forum for identifying which of a set of candidates
is most fit for the office they seek. However, in the current situation, 
debates are often ``my candidate'' versus ``their candidate.'' 
In general, our culture largely adopts the conceptual metaphor \textsc{politics
is war}. But I say we should work towards a world where politics is not war, 
but a collaborative work of art~\cite[p. 139]{Lakoff1980}.
